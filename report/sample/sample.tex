\documentclass[a4paper,11pt]{jsarticle}
\usepackage[dvipdfmx]{graphicx}
\usepackage{url}
\usepackage{booktabs}
\usepackage{listings}

\title{Webプログラミング課題 仕様書レポート}
\author{学籍番号 氏名}
\date{\today}

\begin{document}

\maketitle

\section{利用者向け仕様書}
本セクションでは,開発した3つのWebアプリケーションのうち,代表的なシステムである「サーモンラン敵マニュアル」について,その機能と利用方法を解説する.

\subsection{システム概要}
「サーモンラン敵マニュアル」は,ゲーム『スプラトゥーン』の協力モード「サーモンラン」に登場する敵キャラクター(シャケ)の情報を管理・閲覧するためのWebアプリケーションである.利用者は,敵キャラクターの名前,種類,攻撃方法,討伐方法などの詳細情報を確認できるほか,新しい敵情報の登録や,既存情報の修正・削除を行うことができる.

\subsection{主な機能}
本システムは以下の機能を提供する.

\begin{enumerate}
    \item \textbf{一覧表示機能}: 登録されている敵キャラクターの一覧を表示する.各キャラクターのリンクをクリックすることで詳細画面へ遷移できる.
    \item \textbf{詳細表示機能}: 特定の敵キャラクターに関する詳細な情報(種類,攻撃方法,正規討伐方法,別途討伐方法など)を表示する.
    \item \textbf{新規登録機能}: 新しい敵キャラクターの情報をシステムに追加する.
    \item \textbf{編集機能}: 既存の敵キャラクターの情報を変更する.
    \item \textbf{削除機能}: 不要になった敵キャラクターの情報をシステムから削除する.
\end{enumerate}

\subsection{操作方法}
\subsubsection{情報の閲覧}
トップページ(一覧画面)には,現在登録されている敵キャラクターのIDと名前がリスト形式で表示されている.詳細を知りたいキャラクターの名前(リンク)を選択すると,そのキャラクターの「詳細画面」が表示される.

\subsubsection{情報の追加}
一覧画面にある「追加」リンクを選択すると,「新規登録画面」へ遷移する.入力フォームに以下の項目を入力し,「登録」ボタンを押下することで,新しいデータが保存され一覧画面に反映される.
\begin{itemize}
    \item 名前
    \item 種類(雑魚シャケ,オオモノシャケなど)
    \item 攻撃方法
    \item 正規討伐方法
    \item 別途討伐方法
\end{itemize}

\subsubsection{情報の編集・削除}
「詳細画面」の下部にある「編集」リンクを選択すると,登録内容を変更できる.また,「削除」リンクを選択すると,確認ダイアログが表示され,承認するとそのデータが削除される.

\section{管理者向け仕様書}
本セクションでは,本サービス全体(3つのシステムを含む)の管理方針とシステムの全体像について記述する.

\subsection{サービス全体概要}
本サービスは,特定のテーマに基づいたデータ管理機能(CRUD)を提供する3つの独立したWebアプリケーション群で構成されている.
\begin{enumerate}
    \item \textbf{サーモンラン敵マニュアル}: ゲームキャラクターのデータ管理
    \item \textbf{第五人格ハンター一覧}: ゲームキャラクターのデータ管理
    \item \textbf{D.LEAGUE参画チーム一覧}: ダンスリーグチームのデータ管理
\end{enumerate}
これらは共通の設計思想に基づき実装されており,管理者は統一された操作感で各データをメンテナンスすることが可能である.

\subsection{データの管理と制約}
本システムは学習用プロトタイプとして設計されているため,以下の運用上の制約が存在する.
\begin{itemize}
    \item \textbf{データ永続性の欠如}: データベースを使用せず,サーバーのメモリ上(変数)にデータを保存している.そのため,\textbf{サーバープログラムを再起動または停止すると,追加・編集・削除したデータはすべて初期状態にリセットされる}.
    \item \textbf{同時アクセス}: 簡易的な実装であるため,多数のユーザーによる同時書き込みが発生した場合の排他制御は行われていない.
\end{itemize}

\section{開発者向け仕様書}
本セクションでは,実装された3つのシステムそれぞれの技術仕様について記述する.全てのシステムは Node.js 環境上で動作し,Webフレームワークとして Express,テンプレートエンジンとして EJS を使用している.また,データストアはオンメモリの配列変数を使用している.

\subsection{共通仕様}
すべてのシステムにおいて,以下の設計方針が統一されている.
\begin{itemize}
    \item \textbf{アーキテクチャ}: MVCモデル(Modelは配列変数で代用)に準じた構成.
    \item \textbf{HTTPメソッド}: GET(表示,フォーム取得),POST(登録,更新)を使用.削除は簡易的にGETメソッドで実装されている.
    \item \textbf{JavaScriptモード}: \texttt{"use strict";} モードを使用.
\end{itemize}

\subsection{システムA: サーモンラン敵マニュアル}
\subsubsection{データ構造}
[cite_start]データはサーバーサイドの変数 \texttt{samochara}(配列)に格納される.各要素は以下のオブジェクト構造を持つ [cite: 3, 4].
\begin{table}[h]
    \centering
    \caption{サーモンラン敵マニュアル データ構造}
    \begin{tabular}{llp{8cm}} \toprule
        プロパティ名 & 型 & 説明 \\ \midrule
        id & Number & システム内部で管理されるユニークID \\
        code & String & 編集フォーム等で使用される予備ID(任意入力) \\
        name & String & キャラクター名(例:バクダン) \\
        kind & String & 種類(例:オオモノシャケ) \\
        attack & String & 攻撃方法の説明 \\
        deth & String & 正規の討伐方法 \\
        power & String & 別途討伐方法や特殊な攻略法 \\ \bottomrule
    \end{tabular}
\end{table}

\subsubsection{リソースとエンドポイント}
\begin{itemize}
    \item \textbf{一覧表示}: \texttt{GET /samoran} \\
    \texttt{samoran.ejs} をレンダリングし,全データをリスト表示する.
    \item \textbf{新規登録フォーム}: \texttt{GET /samoran/create} \\
    静的ファイル \texttt{public/samoran\_new.html} へリダイレクトする.
    \item \textbf{詳細表示}: \texttt{GET /samoran/:number} \\
    配列のインデックス \texttt{:number} に対応するデータを \texttt{samoran\_detail.ejs} で表示する.
    \item \textbf{新規登録実行}: \texttt{POST /samoran} \\
    フォームデータを配列に追加し,一覧画面を再レンダリングする.IDは配列長に基づき自動採番される.
    \item \textbf{編集フォーム}: \texttt{GET /samoran/edit/:number} \\
    対象データを埋め込んだ状態で \texttt{samoran\_edit.ejs} を表示する.
    \item \textbf{更新実行}: \texttt{POST /samoran/update/:number} \\
    配列の該当インデックスのデータをフォームの内容で上書きし,一覧画面へリダイレクトする.
    \item \textbf{削除実行}: \texttt{GET /samoran/delete/:number} \\
    配列から該当要素を \texttt{splice} で削除し,一覧画面へリダイレクトする.
\end{itemize}

\subsection{システムB: 第五人格ハンター一覧}
\subsubsection{データ構造}
[cite_start]データは変数 \texttt{hunter}(配列)に格納される.各要素の構造は以下の通りである [cite: 7, 8].
\begin{table}[h]
    \centering
    \caption{第五人格ハンター一覧 データ構造}
    \begin{tabular}{llp{7cm}} \toprule
        プロパティ名 & 型 & 説明 \\ \midrule
        id & Number & ユニークID \\
        name & String & キャラクター名(ハンター名) \\
        real & String & 本名 \\
        birth & String & 誕生日 \\
        weapon & String & 使用武器 \\
        subw & String & 別途攻撃武器・スキル \\
        str & String & キャラクターの強み \\
        less & String & キャラクターの弱点 \\
        dif & Number/String & 操作難易度(最大3) \\ \bottomrule
    \end{tabular}
\end{table}

\subsubsection{リソースとエンドポイント}
基本構造はシステムAと同様であるが,ベースURLが \texttt{/daigo} となる.
\begin{itemize}
    \item 一覧: \texttt{GET /daigo}
    \item 詳細: \texttt{GET /daigo/:number}
    \item 登録: \texttt{POST /daigo}
    \item 更新: \texttt{POST /daigo/update/:number}
    \item 削除: \texttt{GET /daigo/delete/:number}
\end{itemize}

\subsection{システムC: D.LEAGUE参画チーム一覧}
\subsubsection{データ構造}
[cite_start]データは変数 \texttt{teams}(配列)に格納される.各要素の構造は以下の通りである [cite: 11, 12].
\begin{table}[h]
    \centering
    \caption{D.LEAGUE参画チーム一覧 データ構造}
    \begin{tabular}{llp{7cm}} \toprule
        プロパティ名 & 型 & 説明 \\ \midrule
        id & Number & ユニークID \\
        name & String & チーム名 \\
        join & String & 参画シーズン \\
        main & String & メインジャンル(Hiphop等) \\
        people & String & 在籍人数 \\
        win & String & 優勝履歴 \\
        url & String & チーム公式サイトのURL \\ \bottomrule
    \end{tabular}
\end{table}

\subsubsection{リソースとエンドポイント}
基本構造は他システムと同様であり,ベースURLは \texttt{/dleague} となる.
\begin{itemize}
    \item 一覧: \texttt{GET /dleague}
    \item 詳細: \texttt{GET /dleague/:number} (詳細画面では \texttt{url} プロパティを利用した外部リンクが生成される)
    \item 登録: \texttt{POST /dleague}
    \item 更新: \texttt{POST /dleague/update/:number}
    \item 削除: \texttt{GET /dleague/delete/:number}
\end{itemize}

\end{document}